%!TEX TS-program = xelatex
\documentclass[]{friggeri-cv}
\usepackage{fontawesome}
\usepackage{tikz}
\usetikzlibrary{decorations.markings}
\newfontfamily{\FA}{FontAwesome Regular}
\addbibresource{publications.bib}
\def\twitter{{\FA \faTwitter}}
\def\github{{\FA \faGithub}}
\def\linkedin{{\FA \faLinkedin}}
\def\envelope{{\FA \faEnvelopeAlt}}
\def\phone{{\FA \faPhone}}
\def\mobilePhone{{\FA \faMobilePhone}}
\def\book{{\FA \faBook}}
\def\flask{{\FA \faFlask}}
\def\users{{\FA \faUsers}}
\def\pencil{{\FA \faPencil}}
\def\suitcase{{\FA \faSuitcase}}
\def\quotesymbol{{\FA \faQuote}}

\newcommand{\printaside}{
  \begin{aside}
    \section{About}
      \\
      {6 rue de Chateaudun\\
        94200 Ivry-sur-Seine\\
        France\\
        ~\\
        \phone\ 0178542907\\
        \mobilePhone\ 0615231394\\
        ~\\
        \href{mailto:andresrommier@gmail.com}{\envelope andresrommier@gmail.com}\\
        \href{http://twitter.com/andresrommier}{\twitter\ andresrommmier}\\
        \href{http://andresromero.github.com}{\github\ andresromero}\\
        \href{https://www.linkedin.com/in/romeromier}{\linkedin\ romeromier}}\\
    \section{Languages}\\
      {Full working proficiency in spanish, english and french}\\
    \section{Programming/ \\ architectures}\\
      {C and C++ (12+ years), Matlab (8+ years), Python (4+ years), PHP (4+ years), Javascript, {\stdfont \LaTeX} }\\
      ~\\
    % \section{parallel architectures}\\
      {OpenMP, SIMD (SSE, AVX, Neon and Altivec)}\\
      ~\\
      {Notions of Massive Data Mining using MapReduce (Hadoop) and R}\\
      ~\\
    % \section{DSP's}\\
      {TI C6000 and C5000}\\
    % \section{embedded systems}\\
      {Panda boards, Raspberry PI's and Intel ULV platforms}\\
    \section{Operating systems}\\
      {Strong Linux/Unix skills, Mac OS X, Windows}\\
  \end{aside}
}

\begin{document}
\header{\thinfont Andrés\ }{\headingfont Romero}
       {PhD computer vision and parallel architectures}


% In the aside, each new line forces a line break
\section{{\flask}\ Interests}

My research focuses on the development and acceleration of signal processing and computer vision algorithms. A key goal of my research is to develop high performance algorithms for \textbf{visual object detection}, \textbf{tracking} and \textbf{texture description} for real-world problems. The algorithms I have worked on require the understanding of a broad set of areas: \textbf{machine learning}, \textbf{software engineering}, \textbf{computer architecture} and \textbf{statistics}.

In total, I have more than 8 years of research experience working with signal and image processing algorithms in embedded architectures. I am familiar with \textbf{ARM architectures} (ARM Cortex A9 and A15) and the \textbf{Texas Instruments C6000 DSP} family. On those platforms I have implemented many image, voice and audio processing algorithms.

Other fields of interest to me are \textbf{data science}, \textbf{natural language processing} (NLP), \textbf{data mining}, \textbf{information theory}, \textbf{complexity theory} and \textbf{cybernetics}.

\printaside

\section{{\pencil}\ Education}

\begin{entrylist}
  \entry
    {2010-2013}
    {Ph.D. {\normalfont in Computer Science}}
    {\href{http://www.lri.fr}{Laboratoire de Recherche en Informatique}, Université Paris Sud}
    {\textbf{Advisors}: \href{https://www.lri.fr/~lacas/}{\mycolor{orange}{Lionel Lacassagne}} and \href{http://m.i.c.h.e.l.e.free.fr/}{\mycolor{orange}{Michèle Gouiffès}}\\
    Real-time multi-target tracking: A study on color-texture covariance matrices and descriptor/operator switching. \\
    The results of this work were considered for the \textbf{ITEA/Spy european project}.
    }
  \entry
    {2006–2008}
    {M.Sc. {\normalfont Signal Processing}}
    {Instituto Politécnico Nacional (IPN)}
    {
      \textbf{Advisors}: \mycolor{orange}{Héctor Pérez Meana} and \mycolor{orange}{Mariko Nakano Miyatake} \\
      Subject: Acoustic active noise controllers in TMS320C6713DSK platforms.
    }
  \entry
    {2000-2005}
    {Telecommunications Engineering}
    {Universidad Nacional Autónoma de México (UNAM)}
    {
      \textbf{Advisor}: \mycolor{orange}{Bohumil Psenicka} \\
      Graduation project: Acoustic active noise cancellation.
    }
  \entry
    {2000}
    {High-school}
    {Centro Universitario México, Mexico City}
    {Specialization in mathematics and physics}
\end{entrylist}

\section{{\suitcase}\ Experience}

\begin{entrylist}
  \entry
    {2013-2014}
    {Université Paris Sud, Orsay Faculty of Sciences}
    {Invited researcher and teacher}
    {Research work: Covariance descriptor algorithm implementation on \enquote{Many-core} architectures (Xeon-Phi)}
  \entry
    {2012-2013}
    {ITEA/Spy European Project}
    {R\&D engineer}
    {Tracking and pedestrian re-identification module, project in collaboration between CASSIDEAN, EOLAN, ENSTA and IEF, Université Paris Sud}
  \entry
    {2009}
    {Minalum de México S.A. de C.V.}
    {R\&D engineer}
    {Brush-less DC motor controller development using PIC's}
  \entry
    {2009}
    {Czech Technical University in Prague, Czech Republic}
    {Visiting Student}
    {Hosted by Prof. Pavel Saradnik}
  \entry
    {2007-2009}
    {Computer security department, DGSCA, UNAM}
    {Software engineer}
    {Emergency response team (CERT) system development.}
\end{entrylist}

\begin{entrylist}
  \entry
    {2006-2007}
    {Operations engineer}
    {Qualcomm, Omnitracs, Mexico City, Mexico}
    {Satellite and GPS surveillance system operator}
  \entry
    {2004}
    {Electoral Institute of Oaxaca, Oaxaca,Mexico}
    {Software engineer}
    {Pre-electoral results system for the 2004 governor and local congress elections}
\end{entrylist}

\printaside

\section{{\users}\ Teaching experience}
\begin{entrylist}
  \entry
    {2013-2014}
    {ATER (Invited researcher and teacher)}
    {Université Paris-Sud, France}
    {
      \textbf{Courses}:
      \begin{itemize}
      \item SIMD instructions for image processing,
      \item On-line data representations (XML, DOM, XPath, XSLT), 
      \item Relational data bases (SQL), 
      \item Advanced C programming, 
      \item Logical component and computer architecture, 
      \item Scilab.
     \end{itemize}
    }
  \entry
    {2012-2013}
    {Laboratory teacher}
    {Université Paris-Sud, France}
    {
      \textbf{Courses}:
        SIMD instructions for image processing.
    }
  \entry
    {2005-2008}
    {Laboratory teacher}
    {Universidad Nacional Autónoma de México, Mexico}
    {
       \textbf{Courses}:
       \begin{itemize}
         \item Digital signal processing algorithm implementations on DSP architectures,
         \item Digital and analog filtering.
       \end{itemize}
    }
\end{entrylist}

\section{{\book}\ Publications}

\printbibsection{article}{Article in peer-reviewed journal}
\begin{refsection}
  \nocite{*}
  \printbibliography[sorting=chronological, type=inproceedings, title={International peer-reviewed conferences/proceedings}, notkeyword={france}, heading=subbibliography]
\end{refsection}
\begin{refsection}
  \nocite{*}
  \printbibliography[sorting=chronological, type=inproceedings, title={local peer-reviewed conferences/proceedings}, keyword={france}, heading=subbibliography]
\end{refsection}
\printbibsection{misc}{other publications}
\printbibsection{report}{research reports}

\section{{\quotesymbol}\ References}
  Lionel Lacassagne, Associate Professor (MCF HDR) \\
  Laboratoire de Recherche en Informatique (LRI)
  Université Paris Sud\\
  \envelope\ lionel.lacassagne@lri.fr\\
  \phone\ +33 0169154124

  \dottedline

  Michèle Gouiffès, Enseignant Chercheur \\
  Laboratoire d'Informatique pour la Mécanique et les Sciences de l'Ingénieur \\
  Université Paris Sud\\
  \envelope\ michele.gouiffes@u-psud.fr\\
  \phone\ +33 0169858113

  \dottedline

  Rubén Aquino Luna,  \\
  Subdirector of Information Security, DGTIC, UNAM \\
  Universidad Nacional Autónoma de México \\
  \envelope\ raquino@seguridad.unam.mx\\
  \phone\ +52 55 56223064

  \dottedline

  Bohumil Psenicka, Full-time definite C-level Professor \\
  Facultad de Ingeniería, UNAM, Mexico City, Mexico \\
  Universidad Nacional Autónoma de México \\
  \envelope\ pseboh@servidor.unam.mx\\
  \phone\ +52 55 56223064

\printaside

\end{document}
